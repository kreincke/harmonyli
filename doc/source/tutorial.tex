% mycsrf cloak file
%
% (c) Karsten Reincke, Frankfurt a.M. 2010, 2011, ff.
%
% This file is licensed under the Creative Commons Attribution 3.0 Germany
% License (http://creativecommons.org/licenses/by/3.0/de/): 
% For details see teh file LICENSE in the top directory
%
% select the document class
% S.26: [ 10pt|11pt|12pt onecolumn|twocolumn oneside|twoside notitlepage|titlepage final|draft
%         leqno fleqn openbib a4paper|a5paper|b5paper|letterpaper|legalpaper|executivepaper openrigth ]
% S.25: { article|report|book|letter ... }
%
% oder koma-skript S.10 + 16
\documentclass[
  DIV=calc,
  BCOR=5mm,
  12pt,
  headings=small,
  oneside,
  abstract=true,
  toc=bib,
  xcolor=dvipsnames,
  openany,
  ngerman,english]{scrartcl}
  
%%% (1) general configurations %%%
\usepackage[utf8]{inputenc}

%%% (2) language specific configurations %%%
\usepackage[]{a4,babel}
\selectlanguage{english}

% package for improving the grey value and the line feed handling
\usepackage{microtype}

%language specific quoting signs
%default for language english is american style of quotes
%\usepackage[english=british]{csquotes}
\usepackage[english=american]{csquotes}

% jurabib configuration
\usepackage[see]{jurabib}
\bibliographystyle{jurabib}
% mycsrf English jurabib configuration include module file  
%
% (c) Karsten Reincke, Frankfurt a.M. 2012, ff.
%
% This text is licensed under the Creative Commons Attribution 3.0 Germany
% License (http://creativecommons.org/licenses/by/3.0/de/): Feel free to share
% (to copy, distribute and transmit) or to remix (to adapt) it, if you respect
% how you must attribute the work in the manner specified by the author(s):
% \newline
% In an internet based reuse please link the reused parts to mycsrf.fodina.de
% and mention the original author Karsten Reincke in a suitable manner. In a
% paper-like reuse please insert a short hint to mycsrf.fodina.de and to the
% original author, Karsten Reincke, into your preface. For normal quotations
% please use the scientific standard to cite.

% the first time cite with all data, later with shorttitle
\jurabibsetup{citefull=first}

%%% (1) author / editor list configuration
%\jurabibsetup{authorformat=and} % uses 'und' instead of 'u.'
% therefore define your own abbreviated conjunction: 
% an 'and before last author explicetly written conjunction

% for authors in citations
\renewcommand*{\jbbtasep}{\ a.\ } % bta = between two authors sep
\renewcommand*{\jbbfsasep}{,\ } % bfsa = between first and second author sep
\renewcommand*{\jbbstasep}{,\ a.\ }% bsta = between second and third author sep
% for editors in citations
\renewcommand*{\jbbtesep}{\ a.\ } % bta = between two authors sep
\renewcommand*{\jbbfsesep}{,\ } % bfsa = between first and second author sep
\renewcommand*{\jbbstesep}{,\ a.\ }% bsta = between second and third author sep

% for authors in literature list
\renewcommand*{\bibbtasep}{\ a.\ } % bta = between two authors sep
\renewcommand*{\bibbfsasep}{,\ } % bfsa = between first and second author sep
\renewcommand*{\bibbstasep}{,\ a.\ }% bsta = between second and third author sep
% for editors  in literature list
\renewcommand*{\bibbtesep}{\ a.\ } % bte = between two editors sep
\renewcommand*{\bibbfsesep}{,\ } % bfse = between first and second editor sep
\renewcommand*{\bibbstesep}{,\ a.\ }% bste = between second and third editor sep

% use: name, forname, forname lastname u. forname lastname
\jurabibsetup{authorformat=firstnotreversed}
\jurabibsetup{authorformat=italic}

%%% (2) title configuration
% in every case print the title, let it be seperated from the 
% author by a colon and use the slanted font
\jurabibsetup{titleformat={all,colonsep}}
%\renewcommand*{\jbtitlefont}{\textit}

%%% (3) seperators in bib data
% separate bibliographical hints and page hints by a comma
\jurabibsetup{commabeforerest}

%%% (4) specific configuration of bibdata in quotes / footnote
% use a.a.O if possible
\jurabibsetup{ibidem=strict}
% replace ugly a.a.O. by translation of ders., a.a.O.
\AddTo\bibsgerman{
  \renewcommand*{\ibidemname}{Id.,\ l.c.}
  \renewcommand*{\ibidemmidname}{id.,\ l.c.}
}
\renewcommand*{\samepageibidemname}{Id.,\ ibid.}
\renewcommand*{\samepageibidemmidname}{id.,\ ibid.}

%%% (5) specific configuration of bibdata in bibliography
% ever an in: before journal and collection/book-tiltes 
\renewcommand*{\bibbtsep}{in:\ }
\renewcommand*{\bibjtsep}{in:\ }


% ever a colon after author names 
\renewcommand*{\bibansep}{:\ }
% ever a semi colon after the title
% \AddTo\bibsgerman{\renewcommand*{\urldatecomment}{Referenzdownload: }}
\renewcommand*{\bibatsep}{;\ }
% ever a comma before date/year
\renewcommand*{\bibbdsep}{,\ }

% let jurabib insert the S. and p. information
% no S. necessary in bib-files and in cites/footcites
\jurabibsetup{pages=format}

% use a compressed literature-list using a small line indent
\jurabibsetup{bibformat=compress}
\setlength{\jbbibhang}{1em}

% which follows the design of the cites and offers comments
\jurabibsetup{biblikecite}

% print annotations into bibliography
\jurabibsetup{annote}
\renewcommand*{\jbannoteformat}[1]{{ \itshape #1 }}

%refine the prefix of url download
\AddTo\bibsgerman{\renewcommand*{\urldatecomment}{reference download: }}

% we want to have the year of articles in brackets
\renewcommand*{\bibaldelim}{(}
\renewcommand*{\bibardelim}{)}


% in english version Nr. must be replaced by No.
\renewcommand*{\artnumberformat}[1]{\unskip,\space No.~#1}
\renewcommand*{\pernumberformat}[1]{\unskip\space No.~#1}%
\renewcommand*{\revnumberformat}[1]{\unskip\space No.~#1}%

%Reformatting Seriestitle and Seriesnumber
\DeclareRobustCommand{\numberandseries}[2]{%
\unskip\unskip%,
\space\bibsnfont{(=~#2}%
\ifthenelse{\equal{#1}{}}{)}{, [Vol./No.]~#1)}%
}%

\usepackage{xpatch}
\AfterFile{enjbbib.ldf}{%
  \xapptocmd{\bibsenglish}{%
     \def\inname{\ifjboxford in:\else\ifjbchicago in:\else in:\fi\fi}%
    \def\incollinname{\ifjboxford in:\else\ifjbchicago in:\else in:\fi\fi}%
  }{}{}%
}

% the field printed before ISBN, ISSN or URL is the bibfield note
% Hence: If you insert into the field note the type of the literature
% [ Print | [FreeWeb | BibWeb] / [ PDF | HTML ] ] then you now
% get entries like:
% Print: ISBN ....
% BibWeb / PDF => http...
% That's nice for dealing with electronic sources correctly
\DeclareRobustCommand{\jbissn}[1]{\unskip:\space ISSN #1}%
\DeclareRobustCommand{\jbisbn}[1]{\unskip:\space ISBN #1}%

\DeclareRobustCommand{\biburlprefix}{$\Rightarrow$ }
\DeclareRobustCommand{\biburlsuffix}{}



% language specific hyphenation
%mycsrf Hyphenation Include Module text
%
% (c) Karsten Reincke, Frankfurt a.M. 2012, ff.
%
% This text is licensed under the Creative Commons Attribution 3.0 Germany
% License (http://creativecommons.org/licenses/by/3.0/de/): Feel free to share
% (to copy, distribute and transmit) or to remix (to adapt) it, if you respect
% how you must attribute the work in the manner specified by the author(s):
% \newline
% In an internet based reuse please link the reused parts to mycsrf.fodina.de
% and mention the original author Karsten Reincke in a suitable manner. In a
% paper-like reuse please insert a short hint to mycsrf.fodina.de and to the
% original author, Karsten Reincke, into your preface. For normal quotations
% please use the scientific standard to cite.
%


\hyphenation{ my-keds there-fo-re addi-tio-nally}




%%% (3) layout page configuration %%%

% select the visible parts of a page
% S.31: { plain|empty|headings|myheadings }
%\pagestyle{myheadings}
\pagestyle{headings}

% select the wished style of page-numbering
% S.32: { arabic,roman,Roman,alph,Alph }
\pagenumbering{arabic}
\setcounter{page}{1}

% select the wished distances using the general setlength order:
% S.34 { baselineskip| parskip | parindent }
% - general no indent for paragraphs
\setlength{\parindent}{0pt}
\setlength{\parskip}{1.2ex plus 0.2ex minus 0.2ex}


%%% (4) general package activation %%%
%\usepackage{utopia}
%\usepackage{courier}
%\usepackage{avant}
\usepackage[dvips]{epsfig}

% graphic

\usepackage{array}
\usepackage{shadow}
\usepackage{fancybox}

\usepackage{amsmath}
\usepackage{amsfonts}
\usepackage{amssymb}
\usepackage{wasysym}

\usepackage{chngcntr}


%- start(footnote-configuration)

\deffootnote[1.5em]{1.5em}{1.5em}{\textsuperscript{\thefootnotemark)\ }}

% if document class = book: count footnotes from start to end
%\counterwithout{footnote}{chapter}
%- end(footnote-configuration)

% package for macking tables with broken lines
\usepackage{multirow}

%for using label as nameref
\usepackage{nameref}

%integrate nomenclature
% mycsrf English Nomenclation Declaration Include Module 
%
% (c) Karsten Reincke, Frankfurt a.M. 2012, ff.
%
% This text is licensed under the Creative Commons Attribution 3.0 Germany
% License (http://creativecommons.org/licenses/by/3.0/de/): Feel free to share
% (to copy, distribute and transmit) or to remix (to adapt) it, if you respect
% how you must attribute the work in the manner specified by the author(s):
% \newline
% In an internet based reuse please link the reused parts to mycsrf.fodina.de
% and mention the original author Karsten Reincke in a suitable manner. In a
% paper-like reuse please insert a short hint to mycsrf.fodina.de and to the
% original author, Karsten Reincke, into your preface. For normal quotations
% please use the scientific standard to cite.
%

\usepackage[intoc]{nomencl}
\let\abbr\nomenclature

\renewcommand{\nomname}{Periodicals, Shortcuts, and Overlapping Abbreviations}
%\renewcommand{\nomname}{Periodika, ihre Kurzformen und generelle Abkürzungen}


% insert point between abbrewviation and explanation
\setlength{\nomlabelwidth}{.20\hsize}
\renewcommand{\nomlabel}[1]{#1 \dotfill}
% reduce the line distance
\setlength{\nomitemsep}{-\parsep}
\makenomenclature


% depth of contents
\setcounter{secnumdepth}{5}
\setcounter{tocdepth}{5}

% Hyperlinks
\usepackage[breaklinks=true]{hyperref}
\hypersetup{bookmarks=true,breaklinks=true,colorlinks=true,citecolor=blue,draft=false}
\newcommand{\acc}[0]{\textit}
\newcommand{\ra}[0]{$\rightarrow$}
\newcommand{\lnka}[1]{\href{#1}{\texttt{#1}}}
\newcommand{\lnkb}[2]{\href{#1}{\texttt{#1} (RDL: #2)}}
\newcommand{\lnkr}[1]{\ra\ \href{#1}{\texttt{#1}}}


\usepackage{harmony}

%\usepackage{bigfoot}
\usepackage{verbatimbox}

\newcommand{\hlyn}[0]{\textit{harmonyli.ly}}
\newcommand{\hlyf}[0]{\texttt{harmonyli.ly}}
\newcommand{\lily}[0]{\textit{LilyPond}}
\newcommand{\has}[1]{\textit{Harmony Analysis Symbol#1}}

\begin{document}

%% use all entries of the bliography
\nocite{*}

%%-- start(titlepage)
\titlehead{Tutorial}
\subject{Release 0.99
}
\title{harmonyli.ly}
\subtitle{Harmonical Analysis Symbols in LilyPond Scores}
\author{Karsten Reincke% mycsrf License Include Module
%
% (c) Karsten Reincke, Frankfurt a.M. 2012, ff.
%
% This LaTeX-File is licensed under the Creative Commons Attribution-ShareAlike
% 3.0 Germany License (http://creativecommons.org/licenses/by-sa/3.0/de/): Feel
% free 'to share (to copy, distribute and transmit)' or 'to remix (to adapt)'
% it, if you '... distribute the resulting work under the same or similar
% license to this one' and if you respect how 'you must attribute the work in
% the manner specified by the author ...':
%
% In an internet based reuse please link the reused parts to www.fodina.de and
% mention the original author Karsten Reincke in a suitable manner. In a
% paper-like reuse please insert a short hint to www.fodina.de and to the
% original author, Karsten Reincke, into your preface. For normal quotations
% please use the scientific standard to cite.
%

\footnote{This text is licensed under the Creative Commons
Attribution-ShareAlike License (CC BY-SA 4.0 =
(https://creativecommons.org/licenses/by-sa/4.0/deed.en): Feel free \glqq{}to
share (to copy, distribute and transmit)\grqq{} or \glqq{}to remix (to
adapt)\grqq{} it, if you \glqq{}[\ldots] distribute the resulting work under the
same or similar license to this one\grqq{} and if you respect how \glqq{}you
must attribute the work in the manner specified by the author [\ldots]\grqq{}):
\newline
In an internet based reuse please link the reused parts to
\texttt{http://www.fodina.de} and mention the original author -- Karsten Reincke
-- in a suitable manner. In a paper-like reuse please expand your preface by a
short hint to \texttt{http://www.fodina.de/} and the original author, Karsten
Reincke. For quotations use the scientific standard to cite.
\newline
{ \tiny \itshape [Based on the scientific framework \texttt{mind your Scholar
Research Framework} \copyright K. Reincke CC BY 3.0 DE http://fodina.de/mycsrf)]
}}
}

%thanks entry cannot be combined with license footnote
%\thanks{den Autoren von KOMA-Script und denen von Jurabib}

\maketitle
%%-- end(titlepage)

\footnotesize
\tableofcontents

\normalsize

\section{Introduction}
\subsection{Some Systematical Preliminary Remarks}
\subsection{Some Historival Preliminary Remarks}

\section{Installation \& Integration}
\begin{itemize}
  \item Clone the \hlyn\ repository or download and extract the \hlyn\ zip
  archive by using the respective (github)
  commands.\footnote{\lnkr{https://github.com/kreincke/harmonyli.ly}}
  \item Copy the file \hlyf\ to any position of your file system from where you
  want to include the \hlyn\ library into your \lily\ file.
  \item Insert the command \texttt{\textbackslash include 
  "YOUR\_PATH\_TO/harmonyli.ly"} into
  you \lily\ file above the first \texttt{score\{\ldots\}} section.
  \item Copy the line \verb|\context{\Lyrics \consists "Text_spanner_engraver"}|
  into your \texttt{layout\{\ldots\}} section
\end{itemize}

\section{Application \& Utilization}

\hlyn\ uses the \lily\ lyrics technique to integrate its \has{s} into a \lily\
score. The benefit is that \lily\ itself aligns the music notes and the
respective symbols and prevents horizontal overlappings, even if the \has is
much longer than the notes.

The very little disadvantage is that you are not totally free to place a \has\
where you normally can insert a \lily\ \texttt{\textbackslash markup}. Instead
of this, \hlyn\ requires a dedicated voice or staff to which you must bind the
row of \has{s} by using the command \texttt{\textbackslash addlyrics}. After
having prepared your \lily\ file as described above you have three opportunities
to do so:

\subsection{Binding \has{s} to a real voice in a staff}

Binding a row of \has{s} to a real staff is straight forward: 

\subsubsection{Result}
\begin{center}
\begin{lilypond}

\version "2.18.2"

\header { tagline = "" }
\include "lilypond/harmonyli.ly"
  
\score {
  \new StaffGroup {
    \time 4/4
    <<
      \new Staff {
        \relative d' {
          \clef "treble"
          \key d \major  
          \stemUp
          < fis a d>2 < fis a dis> < g b e> < g b eis>2 | 
          < fis b fis'>2 < b e gis> < a e' g!> < a d fis>2 \bar "||"
        }   
      }
      \new Staff {
        \relative d { 
          \clef "bass"
          \key d \major  
          \stemDown
          d2 b d cis  |
          d b d4 cis4 d2 \bar "||"
        }   
      }
      \addlyrics {
          \markup \setHas "T" #'(("C"."D")("fr" . " "))
          \markup \setImHas "D" #'(("B"."1")("a" . "7")("fr" . " "))
          \markup \setHas "Sp" #'(("B"."7")("a" . "7")("fl" . " ")("fr" . " "))
          \markup \setHas "D" #'(("T"."x")("B"."3")("a" . "5")("b" . "7")("c" . "♭9>♯8")("fr" . " "))
          \markup \setHas "Tp" #'(("B"."3")("fl" . " ")("fr" . " ")) 
          \markup \setHas "D" #'(("T"."d")("B"."5")("a" . "7")("b" . "8")("fr" . " ")) 
                 
          \initTextSpan "  "
          \markup \openZoomRow "D" #'(("a"."4")("fl" . " "))
          \startTextSpan
          \markup \expZoomRow #'(("a"."3")("fr" . " ")) 
          \stopTextSpan
  
          \markup \setHas "T" #'(("fr" . " "))
        }
    >>
  }
  \layout {
    \context {
      \Lyrics
      \consists "Text_spanner_engraver"
    }
  }
  \midi {}
}
\end{lilypond}
\end{center}

\subsubsection{Code}

\begin{scriptsize}
\begin{verbatim}
\version "2.18.2"
\header { tagline = "" }
\include "harmonyli.ly"
\score {
  \new StaffGroup {
    \time 4/4
    <<
      \new Staff {
        \relative d' {
          \clef "treble" \key d \major \stemUp
          < fis a d>2 < fis a dis> < g b e> < g b eis>2 | 
          < fis b fis'>2 < b e gis> < a e' g!> < a d fis>2 \bar "||"
      } }   

      \new Staff {
        \relative d { 
          \clef "bass" \key d \major \stemDown
          d2 b d cis  | d b d4 cis4 d2 \bar "||"
      } }   
      
      \addlyrics {
          \markup \setHas "T" #'(("C"."D")("fr" . " "))
          \markup \setImHas "D" #'(("B"."1")("a" . "7")("fr" . " "))
          \markup \setHas "Sp" #'(("B"."7")("a" . "7")("fl" . " ")("fr" . " "))
          \markup \setHas "D" #'(("T"."x")("B"."3")
                                ("a" . "5")("b" . "7")("c" . "-9>+8")("fr" . " "))
          \markup \setHas "Tp" #'(("B"."3")("fl" . " ")("fr" . " ")) 
          \markup \setHas "D" #'(("T"."d")("B"."5")("a" . "7")("b" . "8")
                                ("fr" . " "))    
          \initTextSpan "   "
          \markup \openZoomRow "D" #'(("a"."4")("fl" . " "))
          \startTextSpan
          \markup \expZoomRow #'(("a"."3")("fr" . " ")) 
          \stopTextSpan
          \markup \setHas "T" #'(("fr" . " "))
        }
    >>
  }
  \layout { \context{\Lyrics\consists "Text_spanner_engraver"} }
  \midi {}
}
\end{verbatim}
\end{scriptsize}

\subsubsection{Explanation}

This example contains a descant staff and a bass voice and to letter one it
appends the section \texttt{addlyrics}, which contains for each note of the bass
voice one specific \has. If you do not want to use such a 1:1 relation between
notes and \has{s}, you can use one of the other methods.

\subsection{Binding \has{s} to a hidden voice of a real staff}

Binding a row of \has{s} to a (second) hidden voice in a real staff is a bit
tricky: You must rewrite your musical staff as a staff with several voices,
integrate a (mostly very deep) 'artificial' voice into that staff and bind the
symbols to that 'artificial' voice.\footnote{For demonstrating this option, we
colored the 'hidden' voice. If you change the string \texttt{new Voice =
"AnalysisSubline"} into \texttt{new NullVoice = "AnalysisSubline"}, the voice
beoomes really invisble.}

This method is good to be applied in a score with a large amplitude of pitches
(as some romantic piano pieces have): by using it you can enforce a larger
distance between the really used notes and the \has{s}.

\subsubsection{Result}
\begin{center}
\begin{lilypond}
\version "2.18.2"
\include "lilypond/harmonyli.ly"

\paper {
  indent = 0
  ragged-right = ##f
  system-system-spacing #'basic-distance = #20
  score-system-spacing =
    #'((basic-distance . 12)
       (minimum-distance . 6)
       (padding . 1)
       (stretchability . 12))
}

\header { tagline = "" }

global  = { \key d \major  \time 4/4}

descant = \relative c' {
  \clef treble \stemUp \global
  < fis a d>2 < fis a dis> < g b e> < g b eis>2 | 
  < fis b fis'>2 < b e gis> < a e' g!> < a d fis>2 \bar "||"
}

bass = \relative c {
  \clef bass \stemNeutral \global
  d2 b d cis  | d b d4 cis4 d2 \bar "||"
}

hasRhythmHidden =
\relative c, {
  \clef bass \stemDown \global
  \override NoteHead.color = #red
  \override NoteColumn #'ignore-collision = ##t
  c2 c | c c | c c | c4 c4 c2 \bar "||"
}

hasSymbols = \lyricmode {
  \override LyricText.self-alignment-X = #LEFT
  \override LyricExtender.left-padding = #-0.5
  \override LyricExtender.extra-offset = #'(0 . 0.5)

  \markup \setHas "T" #'(("C"."D")("fr" . " "))
  \markup \setImHas "D" #'(("B"."1")("a" . "7")("fr" . " "))
  \markup \setHas "Sp" #'(("B"."7")("a" . "7")("fl" . " ")("fr" . " "))
  \markup \setHas "D" #'(("T"."x")("B"."3")("a" . "5")("b" . "7")
                          ("c" . "♭9>♯8")("fr" . " "))
  \markup \setHas "Tp" #'(("B"."3")("fl" . " ")("fr" . " ")) 
  \markup \setHas "D" #'(("T"."d")("B"."5")("a" . "7")("b" . "8")
                          ("fr" . " "))    
  \initTextSpan "    "
  \markup \openZoomRow "D" #'(("a"."4")("fl" . " "))
  \startTextSpan
  \markup \expZoomRow #'(("a"."3")("fr" . " ")) 
  \stopTextSpan
  \markup \setHas "T" #'(("fr" . " "))
}

\score {
  <<
    \new GrandStaff <<
      \new Staff = upper
      \with { printPartCombineTexts = ##f }
      { <<
          \descant 
        >>
      }
      \new Staff = lower
      \new Voice = "Musical Bass"
      \with { printPartCombineTexts = ##f }
      { <<
          \bass
          % change "Voice" to "NullVoice" to make analyze voice unvisible:
          \new Voice = "AnalysisSubline" {\shiftOff  \hasRhythmHidden}
          \new Lyrics \lyricsto "AnalysisSubline" \hasSymbols
        >>
      }
    >>
  >>
  \layout{ \context { \Lyrics \consists "Text_spanner_engraver" } }
} 
\end{lilypond}
\end{center}

\subsubsection{Code}
\begin{scriptsize}
\begin{verbatim}
\version "2.18.2"
\include "lilypond/harmonyli.ly"

\paper {
  indent = 0
  ragged-right = ##f
  system-system-spacing #'basic-distance = #20
  score-system-spacing =
    #'((basic-distance . 12)
       (minimum-distance . 6)
       (padding . 1)
       (stretchability . 12))
}

\header { tagline = "" }

global  = { \key d \major  \time 4/4}

descant = \relative c' {
  \clef treble \stemUp \global
  < fis a d>2 < fis a dis> < g b e> < g b eis>2 | 
  < fis b fis'>2 < b e gis> < a e' g!> < a d fis>2 \bar "||"
}

bass = \relative c {
  \clef bass \stemNeutral \global
  d2 b d cis  | d b d4 cis4 d2 \bar "||"
}

hasRhythmHidden =
\relative c, {
  \clef bass \stemDown \global
  \override NoteHead.color = #red
  \override NoteColumn #'ignore-collision = ##t
  c2 c | c c | c c | c4 c4 c2 \bar "||"
}

hasSymbols = \lyricmode {
  \override LyricText.self-alignment-X = #LEFT
  \override LyricExtender.left-padding = #-0.5
  \override LyricExtender.extra-offset = #'(0 . 0.5)

  \markup \setHas "T" #'(("C"."D")("fr" . " "))
  \markup \setImHas "D" #'(("B"."1")("a" . "7")("fr" . " "))
  \markup \setHas "Sp" #'(("B"."7")("a" . "7")("fl" . " ")("fr" . " "))
  \markup \setHas "D" #'(("T"."x")("B"."3")("a" . "5")("b" . "7")
                          ("c" . "-9>+8")("fr" . " "))
  \markup \setHas "Tp" #'(("B"."3")("fl" . " ")("fr" . " ")) 
  \markup \setHas "D" #'(("T"."d")("B"."5")("a" . "7")("b" . "8")
                          ("fr" . " "))    
  \initTextSpan "    "
  \markup \openZoomRow "D" #'(("a"."4")("fl" . " "))
  \startTextSpan
  \markup \expZoomRow #'(("a"."3")("fr" . " ")) 
  \stopTextSpan
  \markup \setHas "T" #'(("fr" . " "))
}

\score {
  <<
    \new GrandStaff <<
      \new Staff = upper
      \with { printPartCombineTexts = ##f }
      { <<
          \descant 
        >>
      }
      \new Staff = lower
      \new Voice = "Musical Bass"
      \with { printPartCombineTexts = ##f }
      { <<
          \bass
          % change "Voice" to "NullVoice" to make analyze voice unvisible:
          \new Voice = "AnalysisSubline" {\shiftOff  \hasRhythmHidden}
          \new Lyrics \lyricsto "AnalysisSubline" \hasSymbols
        >>
      }
    >>
  >>
  \layout{ \context { \Lyrics \consists "Text_spanner_engraver" } }
} 
\end{verbatim}
\end{scriptsize}

\subsubsection{Explanation}

This example defines four variables: the right hand voices (\texttt{= descant}),
the left hand voice (\texttt{= bass)}, the hidden voice definining the rhythmic
granularity of the analysis (\texttt{= hasRhythmHidden}) and the respective
stream of \has{s} (\texttt{= hasSymbols}). Inside of the section
\texttt{\textbackslash score{\ldots}} the 'sounding' bass and the 'virtual'
voice \acc{AnalysisSubline} are inserted into the left hand staff. And the
stream of \has{s} \texttt{\textbackslash hasSymbols} is bound to that 'virtual'
voice by using the command \texttt{\textbackslash lyristico} and a reference by
name.\footnote{Due to fact, that the \has{s} are placed under the analysis
staff, it sometimes happens, that next part of your score (after the system
'linefeed') is not sufficiently separated from the system above. Hereby it
becomes difficult to read the score fluently. For increasing or decreasing the
distance between the system lines, you can play around with the values inserted
into the section \texttt{\textbackslash page\{\ldots\}}}


\subsection{Binding \has{s} to a dedicated analysis staff}

Binding a row of \has{s} to a dedicated analysis staff is straight forward
again: you must create a special staff with one voice which only represents the
the rhythm (relevant for the analysis). 

This method is good to be used in a score with many staves. It simplifies to deal
with ignorable passing notes.

\subsubsection{Result}
\begin{center}
\begin{lilypond}

\version "2.18.2"
\include "lilypond/harmonyli.ly"

\paper {
  indent = 0
  ragged-right = ##f
  system-system-spacing #'basic-distance = #20
  score-system-spacing =
    #'((basic-distance . 12)
       (minimum-distance . 6)
       (padding . 1)
       (stretchability . 12))
}

\header { tagline = "" }

global  = { \key d \major  \time 4/4}

descant = \relative c' {
  \clef treble \stemUp \global
  < fis a d>2 < fis a dis> < g b e> < g b eis>2 | 
  < fis b fis'>2 < b e gis> < a e' g!> < a d fis>2 \bar "||"
}

bass = \relative c {
  \clef bass \stemNeutral \global
  d2 b d cis  | d b d4 cis4 d2 \bar "||"
}

hasRhythm = \relative c {
  \stemDown \global
  c2 c | c c | c c | c4 c4 c2 \bar "||"
}

hasSymbols = \lyricmode {
  \override LyricText.self-alignment-X = #LEFT
  \override LyricExtender.left-padding = #-0.5
  \override LyricExtender.extra-offset = #'(0 . 0.5)

  \markup \setHas "T" #'(("C"."D")("fr" . " "))
  \markup \setImHas "D" #'(("B"."1")("a" . "7")("fr" . " "))
  \markup \setHas "Sp" #'(("B"."7")("a" . "7")("fl" . " ")("fr" . " "))
  \markup \setHas "D" #'(("T"."x")("B"."3")("a" . "5")("b" . "7")
                          ("c" . "♭9>♯8")("fr" . " "))
  \markup \setHas "Tp" #'(("B"."3")("fl" . " ")("fr" . " ")) 
  \markup \setHas "D" #'(("T"."d")("B"."5")("a" . "7")("b" . "8")
                          ("fr" . " "))    
  \initTextSpan "    "
  \markup \openZoomRow "D" #'(("a"."4")("fl" . " "))
  \startTextSpan
  \markup \expZoomRow #'(("a"."3")("fr" . " ")) 
  \stopTextSpan
  \markup \setHas "T" #'(("fr" . " "))
}

\score {
  <<
    \new GrandStaff <<
      \new Staff = upper
      \with { printPartCombineTexts = ##f }{\descant}
      \new Staff = lower
      \with { printPartCombineTexts = ##f }{\bass}
    >>
    \new RhythmicStaff = analysis
    \with { printPartCombineTexts = ##f }
    {
      << 
      \new Voice = "AnalysisLine" { \hasRhythm}
      \new Lyrics \lyricsto "AnalysisLine" \hasSymbols
      >>
    }
  >>

  \layout{ \context{\Lyrics\consists "Text_spanner_engraver"}}
} 

\end{lilypond}
\end{center}


\subsubsection{Code}
\begin{scriptsize}
\begin{verbatim}
\version "2.18.2"
\include "lilypond/harmonyli.ly"

\paper {
  indent = 0
  ragged-right = ##f
  system-system-spacing #'basic-distance = #20
  score-system-spacing =
    #'((basic-distance . 12)
       (minimum-distance . 6)
       (padding . 1)
       (stretchability . 12))
}

\header { tagline = "" }

global  = { \key d \major  \time 4/4}

descant = \relative c' {
  \clef treble \stemUp \global
  < fis a d>2 < fis a dis> < g b e> < g b eis>2 | 
  < fis b fis'>2 < b e gis> < a e' g!> < a d fis>2 \bar "||"
}

bass = \relative c {
  \clef bass \stemNeutral \global
  d2 b d cis  | d b d4 cis4 d2 \bar "||"
}

hasRhythm = \relative c {
  \stemDown \global
  c2 c | c c | c c | c4 c4 c2 \bar "||"
}

hasSymbols = \lyricmode {
  \override LyricText.self-alignment-X = #LEFT
  \override LyricExtender.left-padding = #-0.5
  \override LyricExtender.extra-offset = #'(0 . 0.5)

  \markup \setHas "T" #'(("C"."D")("fr" . " "))
  \markup \setImHas "D" #'(("B"."1")("a" . "7")("fr" . " "))
  \markup \setHas "Sp" #'(("B"."7")("a" . "7")("fl" . " ")("fr" . " "))
  \markup \setHas "D" #'(("T"."x")("B"."3")("a" . "5")("b" . "7")
                          ("c" . "+9>-8")("fr" . " "))
  \markup \setHas "Tp" #'(("B"."3")("fl" . " ")("fr" . " ")) 
  \markup \setHas "D" #'(("T"."d")("B"."5")("a" . "7")("b" . "8")
                          ("fr" . " "))    
  \initTextSpan "    "
  \markup \openZoomRow "D" #'(("a"."4")("fl" . " "))
  \startTextSpan
  \markup \expZoomRow #'(("a"."3")("fr" . " ")) 
  \stopTextSpan
  \markup \setHas "T" #'(("fr" . " "))
}

\score {
  <<
    \new GrandStaff <<
      \new Staff = upper
      \with { printPartCombineTexts = ##f }{\descant}
      \new Staff = lower
      \with { printPartCombineTexts = ##f }{\bass}
    >>
    \new RhythmicStaff = analysis
    \with { printPartCombineTexts = ##f }
    {
      << 
      \new Voice = "AnalysisLine" { \hasRhythm}
      \new Lyrics \lyricsto "AnalysisLine" \hasSymbols
      >>
    }
  >>

  \layout{ \context{\Lyrics\consists "Text_spanner_engraver"}}
} 

\end{verbatim}
\end{scriptsize}

\subsubsection{Explanation}

In general, this example follows the ideas of the latter one\footnote{Due to
fact, that the \has{s} are placed under the analysis staff, it often happens,
that next part of your score (after the system 'linefeed') is not sufficiently
separated from the system above. Hereby it becomes difficult to read the score
fluently. For increasing or decreasing the distance between the system lines,
you can play around with the values inserted into the section
\texttt{\textbackslash page\{\ldots\}}}. But it does not integrate the sounding
bass and the 'virtual' analysis voice into the same staff. Instead of this, each
of them gets its own. And again, the stream of \has{s} is linked to the
analysis voice \acc{hasRhythm} by the command \texttt{\textbackslash lyristico}
and a name reference.

\section{Harmonic Constructs}

After having generally discussed how to integrate and use \hlyn, we can now
explain, how a particular \has is generated / inserted by a respective \hlyn
command. \hlyn offers two interfaces, each consting of a set of functions / commands.

\subsection{The Basic Interface of 'harmonyli.ly'}

The basic interface of \hlyn offers seven functions:

\begin{description}
  \item[setHas] :- inserts a \has{}
    \item[setImHas] :- inserts an intermediary \has{}
  \item[openImRow] :- starts a row of intermediary \has{s} with a first \has{}
  \item[closeImRow] :- closes the row of intermediary \has{s} with the last \has{}
  \item[openZoomRow] :- starts the zoom into a \has{} which contains suspended or
  passing notes
  \item[expZoomRow] :- expands the zoom by a next movement of suspended or
  passing notes
  \item[openImZoomRow] :- starts the zoom into a intermediary \has{} which
  contains suspended or passing notes 
  \item[closeImZoomRow] :- stops the zoom
  into a intermediary \has{} which contains suspended or passing notes
\end{description}

By these functions, you can build five types subrows of your stream of
\has{s}\footnote{Note: \hlyn intends to be complete, but not correct. This means
that 
\begin{itemize}
  \item all kind of 'subrows' of \has{s} which are necessary to describe the
  harmonic relationships of real world chords must be formulatable by \hlyn.
  \item not all \has which can be created by \hlyn necessarily describe
  possible chord rows.
\end{itemize}
}

\begin{description}
  \item[setHas+] :- a row of one or more \has{}
  \item[setImHas+] :- a row of one or more intermediary \has{s}
  \item[openImRow,setHas*,closeImRow] :- the opening of an intermediary area
    \item[openZoomRow,expZoomRow+] :- a zoom-in area opened by a specific
  start-zoom-\has and followed by any number of reduced zoom-expansion-\has{}
  \item[openImZoomRow,expZoomRow*,closeImZoomRow] :-  zoom-in area opened by a
  specific start-zoom-\has and followed by any number of reduced
  zoom-expansion-\has{}
\end{description}


\section{Package Content}














% insert the nomenclature here

% mycsrf Deutsch Nomenclation Tokens Include Module 
%
% (c) Karsten Reincke, Frankfurt a.M. 2012, ff.
%
% This text is licensed under the Creative Commons Attribution 3.0 Germany
% License (http://creativecommons.org/licenses/by/3.0/de/): Feel free to share
% (to copy, distribute and transmit) or to remix (to adapt) it, if you respect
% how you must attribute the work in the manner specified by the author(s):
% \newline
% In an internet based reuse please link the reused parts to mycsrf.fodina.de
% and mention the original author Karsten Reincke in a suitable manner. In a
% paper-like reuse please insert a short hint to mycsrf.fodina.de and to the
% original author, Karsten Reincke, into your preface. For normal quotations
% please use the scientific standard to cite

%\abbr[utb]{UTB}{Uni-Taschenbuch}
%\abbr[stw]{stw}{suhrkamp taschenbuch wissenschaft}

\abbr[cf]{cf.}{confer / compare}
\abbr[id]{id.}{idem = latin for 'the same', be it a man, woman or a group\ldots}
\abbr[ibid]{ibid.}{ibidem = latin for 'at the same place'}
\abbr[ifross]{ifross}{Institut für Rechtsfragen der Freien und Open Source Software}
\abbr[lc]{l.c.}{loco citato = latin for 'in the place cited'}
\abbr[wp]{wp.}{webpage / webdocument without any internal (page)numbering}

%% mycsrf English Nomenclation Tokens Include Module 
%
% (c) Karsten Reincke, Frankfurt a.M. 2012, ff.
%
% This text is licensed under the Creative Commons Attribution 3.0 Germany
% License (http://creativecommons.org/licenses/by/3.0/de/): Feel free to share
% (to copy, distribute and transmit) or to remix (to adapt) it, if you respect
% how you must attribute the work in the manner specified by the author(s):
% \newline
% In an internet based reuse please link the reused parts to mycsrf.fodina.de
% and mention the original author Karsten Reincke in a suitable manner. In a
% paper-like reuse please insert a short hint to mycsrf.fodina.de and to the
% original author, Karsten Reincke, into your preface. For normal quotations
% please use the scientific standard to cite
%

\abbr[afda]{AfdA}{Anzeiger für deutsches Altertum}
\abbr[zfda]{ZfdA}{Zeitschrift für deutsches Altertum und deutsche Literatur [ISSN: 00442518]}
\abbr[zfaw]{}{Zeitschrift für Allgemeine Wissenschaftstheorie / Journal for General Philosophy of Science [ISSN: 0044-2216]}

\printnomenclature

% insert the bibliographical data here
\bibliography{bib/literature}

\end{document}
