% mycsrf cloak file
%
% (c) Karsten Reincke, Frankfurt a.M. 2010, 2011, ff.
%
% This file is licensed under the Creative Commons Attribution 3.0 Germany
% License (http://creativecommons.org/licenses/by/3.0/de/): 
% For details see teh file LICENSE in the top directory
%
% select the document class
% S.26: [ 10pt|11pt|12pt onecolumn|twocolumn oneside|twoside notitlepage|titlepage final|draft
%         leqno fleqn openbib a4paper|a5paper|b5paper|letterpaper|legalpaper|executivepaper openrigth ]
% S.25: { article|report|book|letter ... }
%
% oder koma-skript S.10 + 16
\documentclass[
  DIV=calc,
  BCOR=5mm,
  12pt,
  headings=small,
  oneside,
  abstract=true,
  toc=bib,
  xcolor=dvipsnames,
  openany,
  ngerman,english]{scrartcl}
  
%%% (1) general configurations %%%
\usepackage[utf8]{inputenc}

%%% (2) language specific configurations %%%
\usepackage[]{a4,babel}
\selectlanguage{english}

% package for improving the grey value and the line feed handling
\usepackage{microtype}

%language specific quoting signs
%default for language english is american style of quotes
%\usepackage[english=british]{csquotes}
\usepackage[english=american]{csquotes}

% jurabib configuration
\usepackage[see]{jurabib}
\bibliographystyle{jurabib}
% mycsrf English jurabib configuration include module file  
%
% (c) Karsten Reincke, Frankfurt a.M. 2012, ff.
%
% This text is licensed under the Creative Commons Attribution 3.0 Germany
% License (http://creativecommons.org/licenses/by/3.0/de/): Feel free to share
% (to copy, distribute and transmit) or to remix (to adapt) it, if you respect
% how you must attribute the work in the manner specified by the author(s):
% \newline
% In an internet based reuse please link the reused parts to mycsrf.fodina.de
% and mention the original author Karsten Reincke in a suitable manner. In a
% paper-like reuse please insert a short hint to mycsrf.fodina.de and to the
% original author, Karsten Reincke, into your preface. For normal quotations
% please use the scientific standard to cite.

% the first time cite with all data, later with shorttitle
\jurabibsetup{citefull=first}

%%% (1) author / editor list configuration
%\jurabibsetup{authorformat=and} % uses 'und' instead of 'u.'
% therefore define your own abbreviated conjunction: 
% an 'and before last author explicetly written conjunction

% for authors in citations
\renewcommand*{\jbbtasep}{\ a.\ } % bta = between two authors sep
\renewcommand*{\jbbfsasep}{,\ } % bfsa = between first and second author sep
\renewcommand*{\jbbstasep}{,\ a.\ }% bsta = between second and third author sep
% for editors in citations
\renewcommand*{\jbbtesep}{\ a.\ } % bta = between two authors sep
\renewcommand*{\jbbfsesep}{,\ } % bfsa = between first and second author sep
\renewcommand*{\jbbstesep}{,\ a.\ }% bsta = between second and third author sep

% for authors in literature list
\renewcommand*{\bibbtasep}{\ a.\ } % bta = between two authors sep
\renewcommand*{\bibbfsasep}{,\ } % bfsa = between first and second author sep
\renewcommand*{\bibbstasep}{,\ a.\ }% bsta = between second and third author sep
% for editors  in literature list
\renewcommand*{\bibbtesep}{\ a.\ } % bte = between two editors sep
\renewcommand*{\bibbfsesep}{,\ } % bfse = between first and second editor sep
\renewcommand*{\bibbstesep}{,\ a.\ }% bste = between second and third editor sep

% use: name, forname, forname lastname u. forname lastname
\jurabibsetup{authorformat=firstnotreversed}
\jurabibsetup{authorformat=italic}

%%% (2) title configuration
% in every case print the title, let it be seperated from the 
% author by a colon and use the slanted font
\jurabibsetup{titleformat={all,colonsep}}
%\renewcommand*{\jbtitlefont}{\textit}

%%% (3) seperators in bib data
% separate bibliographical hints and page hints by a comma
\jurabibsetup{commabeforerest}

%%% (4) specific configuration of bibdata in quotes / footnote
% use a.a.O if possible
\jurabibsetup{ibidem=strict}
% replace ugly a.a.O. by translation of ders., a.a.O.
\AddTo\bibsgerman{
  \renewcommand*{\ibidemname}{Id.,\ l.c.}
  \renewcommand*{\ibidemmidname}{id.,\ l.c.}
}
\renewcommand*{\samepageibidemname}{Id.,\ ibid.}
\renewcommand*{\samepageibidemmidname}{id.,\ ibid.}

%%% (5) specific configuration of bibdata in bibliography
% ever an in: before journal and collection/book-tiltes 
\renewcommand*{\bibbtsep}{in:\ }
\renewcommand*{\bibjtsep}{in:\ }


% ever a colon after author names 
\renewcommand*{\bibansep}{:\ }
% ever a semi colon after the title
% \AddTo\bibsgerman{\renewcommand*{\urldatecomment}{Referenzdownload: }}
\renewcommand*{\bibatsep}{;\ }
% ever a comma before date/year
\renewcommand*{\bibbdsep}{,\ }

% let jurabib insert the S. and p. information
% no S. necessary in bib-files and in cites/footcites
\jurabibsetup{pages=format}

% use a compressed literature-list using a small line indent
\jurabibsetup{bibformat=compress}
\setlength{\jbbibhang}{1em}

% which follows the design of the cites and offers comments
\jurabibsetup{biblikecite}

% print annotations into bibliography
\jurabibsetup{annote}
\renewcommand*{\jbannoteformat}[1]{{ \itshape #1 }}

%refine the prefix of url download
\AddTo\bibsgerman{\renewcommand*{\urldatecomment}{reference download: }}

% we want to have the year of articles in brackets
\renewcommand*{\bibaldelim}{(}
\renewcommand*{\bibardelim}{)}


% in english version Nr. must be replaced by No.
\renewcommand*{\artnumberformat}[1]{\unskip,\space No.~#1}
\renewcommand*{\pernumberformat}[1]{\unskip\space No.~#1}%
\renewcommand*{\revnumberformat}[1]{\unskip\space No.~#1}%

%Reformatting Seriestitle and Seriesnumber
\DeclareRobustCommand{\numberandseries}[2]{%
\unskip\unskip%,
\space\bibsnfont{(=~#2}%
\ifthenelse{\equal{#1}{}}{)}{, [Vol./No.]~#1)}%
}%

\usepackage{xpatch}
\AfterFile{enjbbib.ldf}{%
  \xapptocmd{\bibsenglish}{%
     \def\inname{\ifjboxford in:\else\ifjbchicago in:\else in:\fi\fi}%
    \def\incollinname{\ifjboxford in:\else\ifjbchicago in:\else in:\fi\fi}%
  }{}{}%
}

% the field printed before ISBN, ISSN or URL is the bibfield note
% Hence: If you insert into the field note the type of the literature
% [ Print | [FreeWeb | BibWeb] / [ PDF | HTML ] ] then you now
% get entries like:
% Print: ISBN ....
% BibWeb / PDF => http...
% That's nice for dealing with electronic sources correctly
\DeclareRobustCommand{\jbissn}[1]{\unskip:\space ISSN #1}%
\DeclareRobustCommand{\jbisbn}[1]{\unskip:\space ISBN #1}%

\DeclareRobustCommand{\biburlprefix}{$\Rightarrow$ }
\DeclareRobustCommand{\biburlsuffix}{}



% language specific hyphenation
%mycsrf Hyphenation Include Module text
%
% (c) Karsten Reincke, Frankfurt a.M. 2012, ff.
%
% This text is licensed under the Creative Commons Attribution 3.0 Germany
% License (http://creativecommons.org/licenses/by/3.0/de/): Feel free to share
% (to copy, distribute and transmit) or to remix (to adapt) it, if you respect
% how you must attribute the work in the manner specified by the author(s):
% \newline
% In an internet based reuse please link the reused parts to mycsrf.fodina.de
% and mention the original author Karsten Reincke in a suitable manner. In a
% paper-like reuse please insert a short hint to mycsrf.fodina.de and to the
% original author, Karsten Reincke, into your preface. For normal quotations
% please use the scientific standard to cite.
%


\hyphenation{ my-keds there-fo-re addi-tio-nally}




%%% (3) layout page configuration %%%

% select the visible parts of a page
% S.31: { plain|empty|headings|myheadings }
%\pagestyle{myheadings}
\pagestyle{headings}

% select the wished style of page-numbering
% S.32: { arabic,roman,Roman,alph,Alph }
\pagenumbering{arabic}
\setcounter{page}{1}

% select the wished distances using the general setlength order:
% S.34 { baselineskip| parskip | parindent }
% - general no indent for paragraphs
\setlength{\parindent}{0pt}
\setlength{\parskip}{1.2ex plus 0.2ex minus 0.2ex}


%%% (4) general package activation %%%
%\usepackage{utopia}
%\usepackage{courier}
%\usepackage{avant}
\usepackage[dvips]{epsfig}

% graphic

\usepackage{array}
\usepackage{shadow}
\usepackage{fancybox}

\usepackage{amsmath}
\usepackage{amsfonts}
\usepackage{amssymb}
\usepackage{wasysym}

\usepackage{chngcntr}


%- start(footnote-configuration)

\deffootnote[1.5em]{1.5em}{1.5em}{\textsuperscript{\thefootnotemark)\ }}

% if document class = book: count footnotes from start to end
%\counterwithout{footnote}{chapter}
%- end(footnote-configuration)

% package for macking tables with broken lines
\usepackage{multirow}

%for using label as nameref
\usepackage{nameref}

%integrate nomenclature
% mycsrf English Nomenclation Declaration Include Module 
%
% (c) Karsten Reincke, Frankfurt a.M. 2012, ff.
%
% This text is licensed under the Creative Commons Attribution 3.0 Germany
% License (http://creativecommons.org/licenses/by/3.0/de/): Feel free to share
% (to copy, distribute and transmit) or to remix (to adapt) it, if you respect
% how you must attribute the work in the manner specified by the author(s):
% \newline
% In an internet based reuse please link the reused parts to mycsrf.fodina.de
% and mention the original author Karsten Reincke in a suitable manner. In a
% paper-like reuse please insert a short hint to mycsrf.fodina.de and to the
% original author, Karsten Reincke, into your preface. For normal quotations
% please use the scientific standard to cite.
%

\usepackage[intoc]{nomencl}
\let\abbr\nomenclature

\renewcommand{\nomname}{Periodicals, Shortcuts, and Overlapping Abbreviations}
%\renewcommand{\nomname}{Periodika, ihre Kurzformen und generelle Abkürzungen}


% insert point between abbrewviation and explanation
\setlength{\nomlabelwidth}{.20\hsize}
\renewcommand{\nomlabel}[1]{#1 \dotfill}
% reduce the line distance
\setlength{\nomitemsep}{-\parsep}
\makenomenclature


% depth of contents
\setcounter{secnumdepth}{5}
\setcounter{tocdepth}{5}

% Hyperlinks
\usepackage[breaklinks=true]{hyperref}
\hypersetup{bookmarks=true,breaklinks=true,colorlinks=true,citecolor=blue,draft=false}
\newcommand{\lnka}[1]{\href{#1}{\texttt{#1}}}
\newcommand{\lnkb}[2]{\href{#1}{\texttt{#1} (RDL: #2)}}


\usepackage{harmony}

%\usepackage{bigfoot}
\usepackage{verbatimbox}



\begin{document}

%% use all entries of the bliography
\nocite{*}

%%-- start(titlepage)
\titlehead{Tutorial}
\subject{Release 0.99
}
\title{harmonyli.ly}
\subtitle{Harmonical Analysis Symbols in LilyPond Scores}
\author{Karsten Reincke% mycsrf License Include Module
%
% (c) Karsten Reincke, Frankfurt a.M. 2012, ff.
%
% This LaTeX-File is licensed under the Creative Commons Attribution-ShareAlike
% 3.0 Germany License (http://creativecommons.org/licenses/by-sa/3.0/de/): Feel
% free 'to share (to copy, distribute and transmit)' or 'to remix (to adapt)'
% it, if you '... distribute the resulting work under the same or similar
% license to this one' and if you respect how 'you must attribute the work in
% the manner specified by the author ...':
%
% In an internet based reuse please link the reused parts to www.fodina.de and
% mention the original author Karsten Reincke in a suitable manner. In a
% paper-like reuse please insert a short hint to www.fodina.de and to the
% original author, Karsten Reincke, into your preface. For normal quotations
% please use the scientific standard to cite.
%

\footnote{This text is licensed under the Creative Commons
Attribution-ShareAlike License (CC BY-SA 4.0 =
(https://creativecommons.org/licenses/by-sa/4.0/deed.en): Feel free \glqq{}to
share (to copy, distribute and transmit)\grqq{} or \glqq{}to remix (to
adapt)\grqq{} it, if you \glqq{}[\ldots] distribute the resulting work under the
same or similar license to this one\grqq{} and if you respect how \glqq{}you
must attribute the work in the manner specified by the author [\ldots]\grqq{}):
\newline
In an internet based reuse please link the reused parts to
\texttt{http://www.fodina.de} and mention the original author -- Karsten Reincke
-- in a suitable manner. In a paper-like reuse please expand your preface by a
short hint to \texttt{http://www.fodina.de/} and the original author, Karsten
Reincke. For quotations use the scientific standard to cite.
\newline
{ \tiny \itshape [Based on the scientific framework \texttt{mind your Scholar
Research Framework} \copyright K. Reincke CC BY 3.0 DE http://fodina.de/mycsrf)]
}}
}

%thanks entry cannot be combined with license footnote
%\thanks{den Autoren von KOMA-Script und denen von Jurabib}

\maketitle
%%-- end(titlepage)

\section{Introduction}

\section{Installation}


\section{Integration}

\section{Harmonic Constructs}

\footnotesize
\tableofcontents

\normalsize








\section{LilyPond-Beispiel}
\begin{center}
\begin{lilypond}

\version "2.18.2"

\header { tagline = "" }
\include "lilypond/harmonyli.ly"
  
\score {
  \new StaffGroup {
    \time 4/4
    <<
      \new Staff {
        \relative d' {
          \clef "treble"
          \key d \major  
          \stemUp
          < fis a d>2 < fis a dis> < g b e> < g b eis>2 | 
          < fis b fis'>2 < b e gis> < a e' g!> < a d fis>2 \bar "||"
        }
   
      }
      \new Staff {
        \relative d { 
          \clef "bass"
          \key d \major  
          \stemDown
          d2 b d cis  |
          d b d4 cis4 d2 \bar "||"
        }   
      }
      \addlyrics {
          \markup \setHAS "T" #'(("C"."D")("fr" . " "))
          \markup \setImArea "D" #'(("B"."1")("a" . "7")("fr" . " "))
          \markup \setHAS "Sp" #'(("B"."7")("a" . "7")("fl" . " ")("fr" . " "))
          \markup \setHAS "D" #'(("T"."x")("B"."3")("a" . "5")("b" . "7")("c" . "♭9>♯8")("fr" . " "))
          \markup \setHAS "Tp" #'(("B"."3")("fl" . " ")("fr" . " ")) 
          \markup \setHAS "D" #'(("T"."d")("B"."5")("a" . "7")("b" . "8")("fr" . " ")) 
                 
          \initTextSpan "   "
          \markup \initZoomRow "D" #'(("a"."4")("fl" . " "))
          \startTextSpan
          \markup \expZoomRow #'(("a"."3")("fr" . " ")) 
          \stopTextSpan
  
          \markup \setHAS "T" #'(("fr" . " "))
        }
    >>
  }
  \layout {
    \context {
      \Lyrics
      \consists "Text_spanner_engraver"
    }
  }
  \midi {}
}

\end{lilypond}
\end{center}



% insert the nomenclature here

% mycsrf Deutsch Nomenclation Tokens Include Module 
%
% (c) Karsten Reincke, Frankfurt a.M. 2012, ff.
%
% This text is licensed under the Creative Commons Attribution 3.0 Germany
% License (http://creativecommons.org/licenses/by/3.0/de/): Feel free to share
% (to copy, distribute and transmit) or to remix (to adapt) it, if you respect
% how you must attribute the work in the manner specified by the author(s):
% \newline
% In an internet based reuse please link the reused parts to mycsrf.fodina.de
% and mention the original author Karsten Reincke in a suitable manner. In a
% paper-like reuse please insert a short hint to mycsrf.fodina.de and to the
% original author, Karsten Reincke, into your preface. For normal quotations
% please use the scientific standard to cite

%\abbr[utb]{UTB}{Uni-Taschenbuch}
%\abbr[stw]{stw}{suhrkamp taschenbuch wissenschaft}

\abbr[cf]{cf.}{confer / compare}
\abbr[id]{id.}{idem = latin for 'the same', be it a man, woman or a group\ldots}
\abbr[ibid]{ibid.}{ibidem = latin for 'at the same place'}
\abbr[ifross]{ifross}{Institut für Rechtsfragen der Freien und Open Source Software}
\abbr[lc]{l.c.}{loco citato = latin for 'in the place cited'}
\abbr[wp]{wp.}{webpage / webdocument without any internal (page)numbering}

%% mycsrf English Nomenclation Tokens Include Module 
%
% (c) Karsten Reincke, Frankfurt a.M. 2012, ff.
%
% This text is licensed under the Creative Commons Attribution 3.0 Germany
% License (http://creativecommons.org/licenses/by/3.0/de/): Feel free to share
% (to copy, distribute and transmit) or to remix (to adapt) it, if you respect
% how you must attribute the work in the manner specified by the author(s):
% \newline
% In an internet based reuse please link the reused parts to mycsrf.fodina.de
% and mention the original author Karsten Reincke in a suitable manner. In a
% paper-like reuse please insert a short hint to mycsrf.fodina.de and to the
% original author, Karsten Reincke, into your preface. For normal quotations
% please use the scientific standard to cite
%

\abbr[afda]{AfdA}{Anzeiger für deutsches Altertum}
\abbr[zfda]{ZfdA}{Zeitschrift für deutsches Altertum und deutsche Literatur [ISSN: 00442518]}
\abbr[zfaw]{}{Zeitschrift für Allgemeine Wissenschaftstheorie / Journal for General Philosophy of Science [ISSN: 0044-2216]}

\printnomenclature

% insert the bibliographical data here
\bibliography{bib/literature}

\end{document}
